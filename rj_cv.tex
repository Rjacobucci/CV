% (c) 2002 Matthew Boedicker <mboedick@mboedick.org> (original author) http://mboedick.org
% (c) 2003-2007 David J. Grant <davidgrant-at-gmail.com> http://www.davidgrant.ca
% (c) 2008 Nathaniel Johnston <nathaniel@nathanieljohnston.com> http://www.nathanieljohnston.com
%
%This work is licensed under the Creative Commons Attribution-Noncommercial-Share Alike 2.5 License. To view a copy of this license, visit http://creativecommons.org/licenses/by-nc-sa/2.5/ or send a letter to Creative Commons, 543 Howard Street, 5th Floor, San Francisco, California, 94105, USA.

\documentclass[letterpaper,10pt]{article}
\newlength{\outerbordwidth}
\raggedbottom
\raggedright
\usepackage[svgnames]{xcolor}
\usepackage{framed}
\usepackage{comment}
\usepackage{tocloft}
\usepackage{lipsum}
\usepackage{array}

\pagenumbering{arabic}
%-----------------------------------------------------------
%Edit these values as you see fit

\setlength{\outerbordwidth}{3pt}  % Width of border outside of title bars
\definecolor{shadecolor}{gray}{0.75}  % Outer background color of title bars (0 = black, 1 = white)
\definecolor{shadecolorB}{gray}{0.93}  % Inner background color of title bars


%-----------------------------------------------------------
%Margin setup

\setlength{\evensidemargin}{-0.25in}
\setlength{\headheight}{0in}
\setlength{\headsep}{0in}
\setlength{\oddsidemargin}{-0.25in}
\setlength{\paperheight}{11in}
\setlength{\paperwidth}{8.5in}
\setlength{\tabcolsep}{0in}
\setlength{\textheight}{9.5in}
\setlength{\textwidth}{7in}
\setlength{\topmargin}{-0.3in}
\setlength{\topskip}{0in}
\setlength{\voffset}{0.1in}


%-----------------------------------------------------------
%Custom commands

\newcommand\textbox[1]{%
  \parbox{.333\textwidth}{#1}%
}

\newcommand{\resitem}[1]{\item #1 \vspace{-2pt}}
\newcommand{\resheading}[1]{\vspace{8pt}
  \parbox{\textwidth}{\setlength{\FrameSep}{\outerbordwidth}
    \begin{shaded}
\setlength{\fboxsep}{0pt}\framebox[\textwidth][l]{\setlength{\fboxsep}{4pt}\fcolorbox{shadecolorB}{shadecolorB}{\textbf{\sffamily{\mbox{~}\makebox[6.762in][l]{\large #1} \vphantom{p\^{E}}}}}}
    \end{shaded}
  }\vspace{-5pt}
}
\newcommand{\ressubheading}[4]{
\begin{tabular*}{6.5in}{l@{\cftdotfill{\cftsecdotsep}\extracolsep{\fill}}r}
		\textbf{#1} & #2 \\
		\textit{#3} & \textit{#4} \\
\end{tabular*}\vspace{-6pt}}
%-----------------------------------------------------------


\begin{document}

\begin{tabular*}{7in}{l@{\extracolsep{\fill}}r}
\textbf{\Large Ross Jacobucci} & \textbf{Curriculum Vitae} \\
3620 South McClintock Ave., SGM 501  & jacobucc@usc.edu \\
Los Angeles, California 90089 & rjacobucci.com \\
952-201-9347 & Skype: ross.jacobucci

\end{tabular*}
\\


%%%%%%%%%%%%%%%%%%%%%%%%%%%%%%
\resheading{Education}
%%%%%%%%%%%%%%%%%%%%%%%%%%%%%%

% problem w/ \ressubheading
%
\begin{itemize}
\item
	\ressubheading{University of Southern California (USC)}  {Los Angeles, California}{Ph.D. Psychology} {Expected Spring 2017}
	\begin{itemize}
		\resitem{Area: Quantitative Methods}
	\end{itemize}
\item
\ressubheading{University of Southern California (USC)}  {Los Angeles, California}{M.A. Psychology} {December 2015}
\begin{itemize}
	\resitem{Thesis: Regularized Structural Equation Modeling}
	\resitem{Area: Quantitative Methods}
\end{itemize}
\item
	\ressubheading{University of Northern Iowa (UNI)}{Cedar Falls, Iowa}{M.A. Psychology}{2011-2013}
	\begin{itemize}
		\resitem{Individualized Study: Quantitative Emphasis}
	\end{itemize}

\item
	\ressubheading{Luther College}{Decorah, Iowa}{B.A. Psychology}{Graduated 2010}
	\begin{itemize}
		\resitem{Thesis: The Effect of Mental Imagery on Performance in Golf.}
		\resitem{Cum Laude}
	\end{itemize}
\end{itemize}



%%%%%%%%%%%%%%%%%%%%%%%%%%%%%%
\resheading{Publications}
%%%%%%%%%%%%%%%%%%%%%%%%%%%%%%
\begin{itemize} 
\setlength{\topsep}{0pt}%
\setlength{\leftmargin}{0.1in}%
\setlength{\listparindent}{-0.1in}%
\setlength{\itemindent}{-0.2in}%
\setlength{\parsep}{\parskip}%

\item {\textbf{\large{Published Papers and Abstracts}}}
%
\item[]Ammerman, B. A., \textbf{Jacobucci, R.,}, Kleiman, E. M., Muehlenkamp, J. J., \& McCloskey, M. S. (2016). Development and validation of empirically derived frequency criteria for NSSI disorder using exploratory data mining, \emph{Psychological Assessment.}
%
\item[]\textbf{Jacobucci, R.}, Grimm, K. J., \& McArdle, J. J. (2016). Regularized structural equation modeling, \emph{Structural Equation Modeling, 23}, 555-566. doi:10.1080/10705511.2016.1154793. PMCID: 4937830
%
\item[]\textbf{Jacobucci, R.,} \& McArdle, J. J. (2015). Abstract: Regularized structural equation modeling, \emph{Multivariate Behavioral Research 50}, 736-736. doi:10.1080/00273171.2015.1121125. 
%
\item[]Hayes, T., Usami, S., \textbf{Jacobucci, R.,} \& McArdle, J. J. (2015). Using classification and regression trees (CART) and random forests to analyze attrition in longitudinal data: Results from two simulation studies, \emph{Psychology and Aging, 30}, 911-929. doi:10.1037/pag0000046. PMCID: 4743660
%
\item[]Skaar, N. R., Christ, T. J., \& \textbf{Jacobucci, R.} (2014). Measuring adolescent prosocial and health risk behavior in schools: Initial development of a screening measure. \emph{School Mental Health, 6}, 137-149. doi:10.1007/s12310-014-9123-y
%
\item {\textbf{\large{Manuscripts Under Review}}}
%
%\item[]McArdle, J. J, Bohrnstedt, G. A., Prescott, C. A., \textbf{Jacobucci, R.,} Grimm, K. J., Rebok, G., Lyter, D. A., \& Lapham, S. (Under Review). \emph{A confirmatory examination of the factors of cognition as measured in Project TALENT.} Multivariate Behavior Research.
\item[]\textbf{Jacobucci, R.}, Grimm, K. J., \& McArdle, J. J. (in revision). A comparison of methods for uncovering sample heterogeneity: Structural equation model trees and finite mixture models, \emph{Structural Equation Modeling}.
%
\item[]Ammerman, B. A., \textbf{Jacobucci, R.}, Kleiman, E. M., Uyeji, L., \& McCloskey, M. S. (revision under review). The impact of NSSI age of onset on severity and suicidality. \emph{Suicide and Life Threatening Behavior.}
%
\item[]Ammerman, B. A., \textbf{Jacobucci, R.}, \& McCloskey, M. S. (under review).Using exploratory data mining to identify important predictors of non-suicidal self-injury frequency. \emph{Personality and Individual Differences.}
%
\item[]Serang, S., \textbf{Jacobucci, R.}, Brimhall, K. C., \& Grimm, K. J. (under review). Exploratory mediation analysis via regularization. \emph{Structural Equation Modeling}.
%
\item{\textbf{\large{Manuscripts in Preparation}}}
%
\item[] \textbf{Jacobucci, R.}, Grimm, K. J., \& Zelinski, E. M. (in preparation). The bi-directional relationship between health and cognition in older ages. 
%
\item[] \textbf{Jacobucci, R.}, Grimm, K. J. (in preparation). Comparison of frequentist and Bayesian regularization in structural equation modeling. 
%
\item[] \textbf{Jacobucci, R.}, Serang, S, ... (in preparation). Examination of mode effects in CogUSA survey administration.

\item[]Grimm, K. J., \textbf{Jacobucci, R.}, McArdle, J. J. (in preparation). Big data methods and psychological science. Invited submission for Psycholgocical Science Agenda.

\end{itemize} 

%%%%%%%%%%%%%%%%%%%%%%%%%%%%%%
\resheading{Presentations}
%%%%%%%%%%%%%%%%%%%%%%%%%%%%%%
\begin{itemize} 
	\setlength{\topsep}{0pt}%
	\setlength{\leftmargin}{0.1in}%
	\setlength{\listparindent}{-0.1in}%
	\setlength{\itemindent}{-0.2in}%
	\setlength{\parsep}{\parskip}%
	
\item {\textbf{\large{Invited Talks}}}
	
\item[] \textbf{Jacobucci, R.}, Grimm, K. J. (2016, October). \emph{Psuedo-continuous testing of the latent difference score model}. Paper to be presented at the conference to honor the work of Jack McArdle, Richmond, Virginia.
	
\item {\textbf{\large{Symposium Chair}}}

\item[] \emph{Structural Equation Modeling}. 2016 International Meeting for the Psychometric Society, Asheville, North Carolina.
	
%
\item {\textbf{\large{Oral Presentations}}}

\item[]Khoddam, R., Cho, J., \textbf{Jacobucci, R.}, Prescott, C.,\& Leventhal, A. M. (2017, March) \textit{Using bivariate latent difference score modeling to examine the relationship between conduct problems and cigarette use across the transition to high school}. Abstract submitted for paper presentation at the meeting for Society for Research on Nicotine \& Tobacco in Florence, Italy.


\item[] \textbf{Jacobucci, R.}, Zelinski, E. M. (2017, July). \emph{Exploratory search for heterogeneity in change across older age using structural equation model trees.}. Abstract submitted for paper presentation at the 21st IAGG World Congress of Gerontology and Geriatrics, San Francisco, California.

\item[] \textbf{Jacobucci, R.}, Zelinski, E. M. (2016, November). \emph{The bi-directional relationship between health and cognition in older age}. Paper to be presented at the Gerontological Society of America Annual Meeting, New Orleans, Louisiana.

\item[] \textbf{Jacobucci, R.}, Grimm, K. J., McArdle, J. J. (2016, July). \emph{Comparison of frequentist and Bayesian regularization in structural equation modeling.} Paper presented at the International Meeting for the Psychometric Society, Asheville, North Carolina.
%
\item[]Ammerman, B. A., \textbf{Jacobucci, R.,} Kleiman, E. M., Muehlenkamp, J., J. \& McCloskey, M. S. (2015, November). \emph{The exploration and validation of an empirically derived frequency criteria for NSSI Disorder.} Paper presented at the 49th Annual Convention for the Association for Behavioral and Cognitive Therapies, Chicago, Illinois
%
\item[] \textbf{Jacobucci, R.}, Prindle, J. J., McArdle, J. J. (2015, September). \emph{An Application of Exploratory Data Mining in Project TALENT.} Paper presented at the International Meeting for Intelligence Research Annual Conference, Albuquerque, New Mexico.
%
\item {\textbf{\large{Poster Presentations}}}

Ammerman, B. A., \textbf{Jacobucci, R.}, \& McCloskey, M. S. (June, 2016). The Prediction of Non-Suicidal Self-Injury Frequency Using Random Forests. Poster presented at the 11th Annual Meeting of the International Society for the Study of Self-Injury, Eua Claire, Wisconsin.
%
\item[] \textbf{Jacobucci, R.}, Hayes, T., Zelinski, E. M. (2015, November). \emph{Evaluating Methods for Generalizing from a Convenience Sample.} Poster presented at the 68th Annual Scientific Meeting of the Gerontological Society of America, Orlando, Florida.
%
\item[]Zelinski, E., \textbf{Jacobucci, R.}, Kennison, R., \& Zak, D. (2014, November). \emph{Can a convenience sample produce generalizable results?} Poster session presented at the Gerontological Society of America Annual Scientific Meeting, Washington, DC.
%
\item[]\textbf{Jacobucci, R.}, Williams, J. E., \& Thiruselvam, I. (2013, January). \emph{A confirmatory factor analysis of the short form for the IPIP-NEO five-factor model personality scale.} Poster session presented at the Society for Personality and Social Psychology, New Orleans, Louisiana.
%
\item[]\textbf{Jacobucci, R.}, Brownfield, C., \& Williams, J. E. (2012, December). \emph{Intelligence as a factor in criminal offender risk assessment.} Poster session presented at the International Society for Intelligence Research, San Antonio, Texas.
%
\item []\textbf{Jacobucci, R.}, Williams, J. E., \& Thiruselvam, I. (2012, March). \emph{Convergent validity of two short form measures of the five factor model of personality.} Poster session presented at the Iowa Psychological Association Spring Conference, Ames, Iowa.
 
\end{itemize}


%%%%%%%%%%%%%%%%%%%%%%%%%%%%%%
\resheading{Awards/Honors}
%%%%%%%%%%%%%%%%%%%%%%%%%%%%%%
\begin{center}
	\parbox{6.5in}{\textbf{Ruth L. Kirschstein National Research Service Award}, January 2015 - Present}
	\parbox{6.5in}{Predoctoral trainee on the Multidisciplinary Research Training in Gerontology Grant at USC}
\end{center}
\begin{center}
	\parbox{6.5in}{\textbf{2015 SMEP Conference Travel Award}}
	\parbox{6.5in}{Travel award to present at the 13th Annual Society of Multivariate Experimental Psychology Graduate Student Conference.}
	\parbox{6.5in}{Amount: \$1500}
\end{center}
\begin{center}
	\parbox{6.5in}{\textbf{2015 ISIR Conference Travel Award}}
	\parbox{6.5in}{Travel award to present at the annual meeting of the International Society for Intelligence Research.}
	\parbox{6.5in}{Amount: \$1500}
\end{center}
\begin{center}
	\parbox{6.5in}{\textbf{2014 SMEP Workshop Travel Award}}
	\parbox{6.5in}{Travel award to attend the 2014 APA Advanced Training Institutes on Structural Equation Modeling in Longitudinal Research and Exploratory Data Mining in Behavioral Research. }
	\parbox{6.5in}{Amount: \$1000}
\end{center}
\begin{center}
	\parbox{6.5in}{\textbf{2014 APA Science Directorate Travel Award}}
	\parbox{6.5in}{Travel award to attend the 2014 APA Advanced Training Institutes on Structural Equation Modeling in Longitudinal Research and Exploratory Data Mining in Behavioral Research.}
	\parbox{6.5in}{Amount: \$500}
\end{center}
\begin{center}
	\parbox{6.5in}{\textbf{Intercollegiate Academics Fund Travel Award}, UNI, Fall 2012}
	\parbox{6.5in}{Award allocated to promote and support the presentation of research at an academic conference.}
	\parbox{6.5in}{Amount: \$650}
\end{center}
\begin{center}
	\parbox{6.5in}{\textbf{Graduate Student Opportunity Fund For Travel}, UNI, Fall 2012}
	\parbox{6.5in}{Award to support travel for graduate students presenting at an academic conference.}
	\parbox{6.5in}{Amount: \$250}
\end{center}
\begin{center}
	\parbox{6.5in}{\textbf{College of Social and Behavioral Sciences Graduate Research Award}, UNI, Fall 2012}
	\parbox{6.5in}{Award allocated for the purchase of expenses related to thesis research.}
	\parbox{6.5in}{Amount: \$500}
\end{center}
\begin{center}
	\parbox{6.5in}{\textbf{Graduate College Tuition Scholarship}, UNI,  Fall 2012}
	\parbox{6.5in}{Scholarship based on academic performance.}
\end{center}
\begin{center}
	\parbox{6.5in}{\textbf{Academic All-American}, Golf, 2009, 2010}
	\parbox{6.5in}{An award given out by the Golf Coaches Association of America based on academic and athletic performance.}
\end{center}

%\newpage

%%%%%%%%%%%%%%%%%%%%%%%%%%%%%%
%\resheading{Research Experience}
%%%%%%%%%%%%%%%%%%%%%%%%%%%%%%
%\begin{itemize} 
%\setlength{\topsep}{0pt}%
%\setlength{\leftmargin}{0.1in}
%
%\item \textbf{Research Assistant} {Fall 2014-Present }\\
%Supervisor: Dr. Elizabeth Zelinski, USC\\ 
%I am currently using both structural equation models and statistical learning methods to compare the Long Beach Longitudinal Study to the Health and Retirement Study. Some of the main analyses include using the sample weights in HRS to create sample weights in LBLS as well as using SEM to model cognitive decline in LBLS. My main focus is on evaluating different methods of creating sample weights can make the results of a convenience sample, LBLS, more generalizable and to what degree.

%
%\item \textbf{Research Assistant} {Fall 2013-Present }\\
%Supervisor: Dr. John J. McArdle, USC\\ 
%In working with data from Project TALENT (PT), I have conducted both exploratory and confirmatory factor analyses in R and MPlus to determine the factor structure of 60+ cognitive variables. Using these models, in addition to 2000+ other variables collected in the initial and follow up samples, I have used the semtree package to determine relationship between these covariates and factor model parameters. Because of the complexity of models and size of sample, I have utilized the Center for High-Performance Computing and Communication's (HPCC) supercomputer. Finally, I have used multiple item response theory(IRT) models on item level data from 20 cognitive ability scales to determine the best model.
%
%\item \textbf{Psychometric Expert} {Summer 2013-Spring 2014}
%Supervisor: Dr. John J. McArdle, USC\\ 
%Worked closely with Dr. Jiu-Chiuan Chen and Dr. Yong Cen through the Preventative Medicine department at USC in the modeling and prediction of cognitive decline in two large longitudinal datasets. Responsibilities include using structural equation modeling (SEM) to construct models to predict Alzheimer's disease, item response theory and factor analysis to evaluate the factor structure and precision of multiple cognitive measures, and Monte Carlo simulation to conduct multiple power analyses for grant preparation. I have also built on this work in conducting first and second-order latent growth models, mixture models, among other multivariate techniques. 

%
%\item \textbf{Principal Investigator} {Summer 2012-Spring 2013, \emph{Personality and Cognitive Ability in Civilly Committed Sexual Offenders}}\\
%Supervisor: Dr. John E. Williams, UNI\\
%Conducting group administration of cognitive ability and personality measures at the Civil Commitment Unit for Sexual Offenders in Cherokee, Iowa. Responsibilities included conducting a literature review, designing an original study, coordinating with administrators at the civil commitment unit, administration of ability and self-report measures to patients, entering, cleaning, and analyzing data in SPSS.
%
%\item \textbf{Principal Investigator} {Summer 2012-Fall 2012, \emph{Personality, Cognitive Functioning, Adaptive Behavior, and Criminogenic Needs}}\\
%Supervisor: Dr. John E. Williams, UNI\\
%Conducting archival data collection at the Iowa Medical and Classification Center in 	Coralville, Iowa. Examine cognitive ability, academic achievement, personality, risk assessment, and conviction charges. 
%
%\item \textbf{Principal Investigator} {Fall 2011-Summer 2012, \emph{The Role of Grit, Self-Control, and Conscientiousness in Academic Achievement}}\\
%Supervisor: Dr. John E. Williams, UNI\\
%Conducted group administration of self-report measures of personality and academic achievement in a college sample. Responsibilities included the administration of self-report measures to high school and college populations, entering, cleaning, and analyzing data in SPSS, and writing up in APA style manuscript. 
%
%\item[]{\textbf{\large{Lab Experience}}}
%

%\item \textbf{Research Assistant} {Fall 2012-Summer 2013 }\\
%Supervisor: Mike Whitson, iTracking Research, Inc.\\
%Assist in research design, methodologies, and conducting statistical analyes in R and SPSS. Assembled scripts for analyses, cleaned datafiles, constructed and ran SEM models in AMOS, composed graphs of analyses, and other duties as needed.     
%
%\item \textbf{Research Assistant} {Fall 2012-Summer 2013 }\\
%Supervisor: Dr. Andrew R. Gilpin, UNI\\
%Assist in conducting background research and conceptualizing a Monte Carlo simulation study that examines the effectiveness of basing sample size planning on the results of pilot studies.  
%
%\item \textbf{Research Assistant} {Spring 2012-Spring 2013 }\\
%Supervisor: Dr. Nicole Skaar, UNI\\
%Design and run confirmatory factor analyses in AMOS and R on a measure of risk behavior. Conduct IRT analysis of Iowa Youth Survey data. 
%
%\item \textbf{Lab Coordinator} Summer 2012-Fall 2012, Psychometrics and Personality Lab\\
%Supervisor: Dr. John E. Williams, UNI\\
%Responsibilities include coordinating lab related meetings and activities. Additionally, conduct meetings that teach research methods related to the current lab research. 
%


%%%%%%%%%%%%%%%%%%%%%%%%%%%%%%
\resheading{Teaching Experience}
%%%%%%%%%%%%%%%%%%%%%%%%%%%%%%
\begin{itemize} 
\setlength{\topsep}{0pt}%
\setlength{\leftmargin}{0.1in}%

%%%%%%%%%% Short Version %%%%%%%%%%%%%%

\item {\textbf{\large{Workshops}}}

\item[]Exploratory Data Mining via SEARCH Strategies. 8/15/2016-8/19/2016, University of Michigan, Ann Arbor, MI. Role: Lecturer and lab instructor.
%
\item[]Big Data: Exploratory Data Mining in Behavioral Research, APA Advanced Training Institutes. 6/6/2016-6/10/2016, Arizona State University, Tempe, AZ. Role: Lecturer and lab instructor.
%
\item[]Structural Equation Modeling in Longitudinal Research, APA Advanced Training Institutes. 6/6/2016-6/10/2016, Arizona State University, Tempe, AZ. Role: Lab instructor.
%
\item[]Exploratory Data Mining via SEARCH Strategies. 6/8/2015-6/12/2015, University of Michigan, Ann Arbor, MI. Role: Lab instructor.
%
\item[]Big Data: Exploratory Data Mining in Behavioral Research, APA Advanced Training Institutes. 6/1/2015-6/5/2015, Arizona State University, Tempe, AZ. Role: Lab instructor.

\item {\textbf{\large{Lecturing}}}

\item[] \emph{Exploratory Factor Analysis}, Multivariate Analysis of Behavioral Data, Fall 2016, USC\\
\item[] \emph{SEM Trees}, Exploratory Data Mining via SEARCH Strategies Workshop, Summer 2016\\
\item[] \emph{SEM Trees}, Big Data: Exploratory Data Mining in Behavioral Research, Summer 2016\\
\item[] \emph{The application of SEM Trees in Project TALENT}, Big Data: Exploratory Data Mining in Behavioral Research Workshop, Summer 2014, USC.
\item[] \emph{Graphing in R}, Graduate Consultation and Computer Center Lecture Series, Spring 2014, USC.
\item[] \emph{Item Factor Analysis}, Graduate Consultation and Computer Center Lecture Series, Spring 2014, USC.
\item[] \emph{Factor Indeterminacy}, Workshop in Quantitative Methods -- Mathematical Psychology, Spring 2014, USC\\
\item[] \emph{Estimation in Factor Analysis}, Workshop in Quantitative Methods -- Mathematical Psychology, Spring 2014, USC\\
\item[] \emph{Statistical Learning-Part 3: Advanced Algorithms}, Graduate Consultation and Computer Center Lecture Series, Fall 2014, USC.
\item[] \emph{Statistical Learning-Part 1: Preprocessing}, Graduate Consultation and Computer Center Lecture Series, Fall 2014, USC.
\item[] \emph{Exploratory Factor Analysis}, Introduction to Psych Statistics, Spring 2012, UNI\\



\item {\textbf{\large{Teaching Assistant}}}

Fundamentals of Psychological Measurement, Fall 2014 (graduate: Dr. John J. McArdle, USC)\\
History \& Systems, Spring 2013 (undergraduate: Dr. Andrew R. Gilpin, UNI)\\
Psychology of Music, Spring 2013 (undergraduate: Dr. Carolyn Hildebrandt, UNI)\\
Advanced Graduate Statistics, Fall 2012 (graduate: Dr. Andrew R. Gilpin, UNI)\\
Cognitive \& Intellectual Assessment, Fall 2012 (graduate: Dr. John E. Williams, UNI)\\
Honor's Introducton to Psychology, Fall 2012 (undergraduate: Dr. John E. Williams, UNI)\\
Introduction to Psych Statistics, Spring 2012 (undergraduate: Dr. Andrew R. Gilpin, UNI)\\
Memory \& Language, Fall 2011 (undergraduate: Dr. Jack Yates, UNI)\\
%
\end{itemize}


%%%%%%%%%%%%%%%%%%%%%%%%%%%%%%
\resheading{Software}
%%%%%%%%%%%%%%%%%%%%%%%%%%%%%%
\begin{itemize}
	%
	%\item[]{\textbf{\large{Published}}}
	%
	\item[]\textbf{Jacobucci, R.} (2016). regsem: Performs Regularization on Structural Equation Models (version 0.2.0) [Software]. Available from https://cran.r-project.org/web/packages/index.html
	%
	\item[]\textbf{Jacobucci, R.} (2016). autosem: Performs Specification Search in Structural Equation Models (version 0.1.0) [Software]. Available from https://cran.r-project.org/web/packages/autoSEM/index.html
	%
\end{itemize}


%%%%%%%%%%%%%%%%%%%%%%%%%%%%%%
\resheading{Professional Activities}
%%%%%%%%%%%%%%%%%%%%%%%%%%%%%%
\begin{itemize} 
	\setlength{\topsep}{0pt}%
	\setlength{\leftmargin}{0.1in}%

\item {\textbf{\large{Consulting}}}

\item[] Part of environmental-brain aging psychometrics working group for the Environmental Determinants of Cognitive Aging R01, Summer 2013 - Spring 2014, USC.

\item {\textbf{\large{Advanced Training}}}

\item[]Integrative Analysis of Longitudinal Studies of Aging and Dementia:
Enhancing Replication and Reproducibility, GSA 2015 Pre-Conference Workshop, Orlando, FL

\item[]Structural Equation Modeling in Longitudinal Research, APA Advanced Training Institutes. Summer 2014, Arizona State University, Tempe, AZ

\item[] Big Data: Exploratory Data Mining in Behavioral Research, APA Advanced Training Institutes. Summer 2014, Arizona State University, Tempe, AZ


\item {\textbf{\large{Reviewing}}}

	
Multivariate Behavioral Research\\
SMEP 2016 Graduate Student Conference\\
21st IAGG World Congress of Gerontology and Geriatrics\\

	
	


\item {\textbf{\large{Affiliations}}}

	
	%\item{\large{International Society for Intelligence Research}}
	%\item{\large{Association for Psychological Science}}
Gerontological Society of America\\
	%\item{\large{APA Division 15. Educational Psychology}}
Psychometric Society\\
	
	
\end{itemize}


%%%%%%%%%%%%%%%%%%%%%%%%%%%%%%
%\resheading{Consultation}
%%%%%%%%%%%%%%%%%%%%%%%%%%%%%%
%begin{itemize} 
%\setlength{\topsep}{0pt}%
%\setlength{\leftmargin}{0.1in}

%\item \textbf{iTracking Research, Inc.} {Summer 2013-Fall 2013}\\
%Assisted in the methodological development of the analysis of Parkinson's Disease through the use of eyetracking. Developed scripts in R to organize the data for analysis as well as for the use of machine learning algorithms to predict outcomes. 
%
%end{itemize}






%%%%%%%%%%%%%%%%%%%%%%%%%%%%%%
%\resheading{Skills}
%%%%%%%%%%%%%%%%%%%%%%%%%%%%%%

%begin{itemize} 
%\setlength{\topsep}{0pt}%
%\setlength{\leftmargin}{0.1in}%
%
%\item \textbf{Statistical Software}\\
%R,  Mplus, Winsteps, flexMIRT, SAS, SPSS, Amos
%

%\item \textbf{Assessment Administration through Coursework} {Fall 2011-Spring 2012}\\
%Trained on the following measures: Minnesota Multiphasic Personality Inventory–	-Second Edition, Revised Form, and Adolescent Form; Personality Assessment Inventory–-Adult and Adolescent Forms; Structured Clinical Interview for DSM Disorders; Structured Clinical Interview for DSM Personality Disorders; Weschler’s Adult Intelligence Scale–-Fourth Edition; and Woodcock Johnson Achievement Test–-Third Edition.
%
%end{itemize}

%%%%%%%%%%%%%%%%%%%%%%%%%%%%%%
\resheading{References}
%%%%%%%%%%%%%%%%%%%%%%%%%%%%%%

   \begin{center}
    \parbox{6.5in}{\textbf{Elizabeth Zelinski}, Ph.D., University of Southern California}
    \parbox{6.5in}{Professor of Gerontology and Psychology}
    \parbox{6.5in}{zelinski@usc.edu}
    \parbox{6.5in}{Phone: (213) 740-4918}
    \end{center}
      \begin{center}
      	\parbox{6.5in}{\textbf{Kevin J. Grimm}, Ph.D., Arizona State University}
      	\parbox{6.5in}{Professor of Psychology}
      	\parbox{6.5in}{kjgrimm@asu.edu}
      	\parbox{6.5in}{Phone: (480) 965-5946 }
      \end{center} 
        \begin{center}
        	\parbox{6.5in}{\textbf{John J. McArdle}, Ph.D., University of Southern California}
        	\parbox{6.5in}{Professor of Psychology and Gerontology}
        	\parbox{6.5in}{jack\_mcardle@hotmail.com}
        	\parbox{6.5in}{Phone: (213) 740-2276 }
        \end{center} 
%\begin{center}
%  \parbox{6.5in}{\textbf{Mike Whitson},  iTracking Research, Inc.}
%  \parbox{6.5in}{Vice President}
%  \parbox{6.5in}{mike@itrackingresearch.com}
%  \parbox{6.5in}{Phone: (319) 239-9008}
%  \end{center}
%\begin{center}
%  \parbox{6.5in}{\textbf{Nicole Skaar}, Ph.D., University of Northern Iowa}
%  \parbox{6.5in}{Assistant Professor of School Psychology}
%  \parbox{6.5in}{nicole.skaar@uni.edu}
%  \parbox{6.5in}{Phone: 319-273-7649}
%  \end{center}



\end{document}