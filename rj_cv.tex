% (c) 2002 Matthew Boedicker <mboedick@mboedick.org> (original author) http://mboedick.org
% (c) 2003-2007 David J. Grant <davidgrant-at-gmail.com> http://www.davidgrant.ca
% (c) 2008 Nathaniel Johnston <nathaniel@nathanieljohnston.com> http://www.nathanieljohnston.com
%
%This work is licensed under the Creative Commons Attribution-Noncommercial-Share Alike 2.5 License. To view a copy of this license, visit http://creativecommons.org/licenses/by-nc-sa/2.5/ or send a letter to Creative Commons, 543 Howard Street, 5th Floor, San Francisco, California, 94105, USA.

\documentclass[letterpaper,10pt]{article}
\newlength{\outerbordwidth}
\raggedbottom
\raggedright
\usepackage[svgnames]{xcolor}
\usepackage{framed}
\usepackage{comment}
\usepackage{tocloft}
\usepackage{lipsum}
\usepackage{array}
\usepackage{etaremune}

\pagenumbering{arabic}
%-----------------------------------------------------------
%Edit these values as you see fit

\setlength{\outerbordwidth}{3pt}  % Width of border outside of title bars
\definecolor{shadecolor}{gray}{0.75}  % Outer background color of title bars (0 = black, 1 = white)
\definecolor{shadecolorB}{gray}{0.93}  % Inner background color of title bars


%-----------------------------------------------------------
%Margin setup

\setlength{\evensidemargin}{-0.25in}
\setlength{\headheight}{0in}
\setlength{\headsep}{0in}
\setlength{\oddsidemargin}{-0.25in}
\setlength{\paperheight}{11in}
\setlength{\paperwidth}{8.5in}
\setlength{\tabcolsep}{0in}
\setlength{\textheight}{9.5in}
\setlength{\textwidth}{7in}
\setlength{\topmargin}{-0.3in}
\setlength{\topskip}{0in}
\setlength{\voffset}{0.1in}


%-----------------------------------------------------------
%Custom commands

\newcommand\textbox[1]{%
  \parbox{.333\textwidth}{#1}%
}

\newcommand{\resitem}[1]{\item #1 \vspace{-2pt}}
\newcommand{\resheading}[1]{\vspace{8pt}
  \parbox{\textwidth}{\setlength{\FrameSep}{\outerbordwidth}
    \begin{shaded}
\setlength{\fboxsep}{0pt}\framebox[\textwidth][l]{\setlength{\fboxsep}{4pt}\fcolorbox{shadecolorB}{shadecolorB}{\textbf{\sffamily{\mbox{~}\makebox[6.762in][l]{\large #1} \vphantom{p\^{E}}}}}}
    \end{shaded}
  }\vspace{-5pt}
}
\newcommand{\ressubheading}[4]{
\begin{tabular*}{6.5in}{l@{\cftdotfill{\cftsecdotsep}\extracolsep{\fill}}r}
		\textbf{#1} & #2 \\
		\textit{#3} & \textit{#4} \\
\end{tabular*}\vspace{-6pt}}
%-----------------------------------------------------------


\begin{document}

\begin{tabular*}{7in}{l@{\extracolsep{\fill}}r}
\textbf{\Large Ross Jacobucci} & \textbf{Curriculum Vitae} \\
390 Corbett Family Hall  & rjacobuc@nd.edu \\
Notre Dame, Indiana 46556 & rjacobucci.com \\
574-631-2589 & \today

\end{tabular*}
\\


%%%%%%%%%%%%%%%%%%%%%%%%%%%%%%
\resheading{Professional History \& Education}
%%%%%%%%%%%%%%%%%%%%%%%%%%%%%%

% problem w/ \ressubheading
\begin{itemize}
	\item
	\ressubheading{Assistant Professor of Psychology}{2017 - Current} {University of Notre Dame}  {Notre Dame, Indiana}
	\vspace{.15cm}
%
\item
	\ressubheading{Ph.D. Psychology} {August 2017}{University of Southern California (USC)}  {Los Angeles, California}
	\begin{itemize}
		\resitem{Area: Quantitative Methods}
	%	\resitem{Dissertation: Evaluating the Stability of Decision Tree Algorithms.}
	\end{itemize}
%\item
%\ressubheading{M.A. Psychology} {December 2015}{University of Southern California (USC)}  {Los Angeles, California}
%\begin{itemize}
%	\resitem{Area: Quantitative Methods}
%	\resitem{Thesis: Regularized Structural Equation Modeling.}
%\end{itemize}
\item
	\ressubheading{M.A. Psychology}{2011-2013}{University of Northern Iowa (UNI)}{Cedar Falls, Iowa}
	\begin{itemize}
		\resitem{Individualized Study: Quantitative Emphasis}
	\end{itemize}

\item
	\ressubheading{B.A. Psychology}{June 2010}{Luther College}{Decorah, Iowa}
%	\begin{itemize}
	%	\resitem{Thesis: The Effect of Mental Imagery on Performance in Golf.}
	%	\resitem{Cum Laude}
%	\end{itemize}
\end{itemize}


%%%%%%%%%%%%%%%%%%%%%%%%%%%%%%
\resheading{Research Funding}
%%%%%%%%%%%%%%%%%%%%%%%%%%%%%%

\begin{itemize} 
	\setlength{\topsep}{0pt}%
	\setlength{\leftmargin}{0.1in}%
	\setlength{\listparindent}{-0.1in}%
	\setlength{\itemindent}{-0.2in}%
	\setlength{\parsep}{\parskip}%
	
	\item {\textbf{\large{Ongoing}}}
	
	\begin{center}
			\noindent\parbox{0.7\textwidth}{\raggedright \textbf{NIH R21 MH129688-01}}\hfill%
		\parbox{0.2\textwidth}{\raggedright 3/4/22-2/29/24}
		\parbox{6.5in}{{Brooke Ammerman (Co-PI) \& Ross Jacobucci (Co-PI) }  }
	%	\parbox{6.5in} {Impact/Priority Score of 30 (10\%)}
		\parbox{6.5in}{\textit{Advancing Real-Time Suicide Risk Detection Through the Digital Phenotyping Smartphone Application Screenomics}}
		\parbox{6.5in}{Total Costs: \$430,375}
	\end{center}

\begin{center}
	\noindent\parbox{0.7\textwidth}{\raggedright \textbf{NIH R21 MH131978-01}}\hfill%
	\parbox{0.2\textwidth}{\raggedright 12/12/22-11/30/24}
	\parbox{6.5in}{{Ross Jacobucci (Co-PI) \& Brooke Ammerman (Co-PI)}  }
	\parbox{6.5in}{\textit{Improving Momentary Suicide Risk Identification Through Adaptive Time Sampling}}
	\parbox{6.5in}{Total Costs: \$430,375}
\end{center}



\begin{center}
	\noindent\parbox{0.7\textwidth}{\raggedright \textbf{NIH TRA RFA-RM-21-017}}\hfill%
	\parbox{0.2\textwidth}{\raggedright 7/1/2022-6/30/2027}
	\parbox{6.5in}{{Theodore Beauchaine (PI)} }
	\parbox{6.5in} {Role: Co-I}
	\parbox{6.5in}{\textit{Leveraging Noninvasive Transcutaneous Vagus Nerve Stimulation and Smartphone Technology to Reduce Suicidal Behaviors and Suicide Among Highly Vulnerable Adolescents}}
	\parbox{6.5in}{Total Direct: \$ 2,995,095 }
\end{center}

\begin{center}
		\noindent\parbox{0.7\textwidth}{\raggedright \textbf{NIH AIM-AHEAD Pilot Program}}\hfill%
	\parbox{0.2\textwidth}{\raggedright 9/1/2022-9/1/2023}
	\parbox{6.5in}{{Allen Tien (PI)} }
	\parbox{6.5in} {Role: Consultant}
	\parbox{6.5in}{\textit{Establishing student project capabilities to improve suicide prevention (PROTECT): Rich data access, ML algorithm optimization, translation to practice (Model C)}}
	\parbox{6.5in}{Total Direct: \$ 495,000}
\end{center}




\begin{center}
		\noindent\parbox{0.7\textwidth}{\raggedright \textbf{ND Lucy Family Institute for Data \& Society}}\hfill%
\parbox{0.2\textwidth}{\raggedright 6/1/2022 - 5/31/2023}
	\parbox{6.5in}{{Ross Jacobucci (PI)}  }
	%	\parbox{6.5in} {Impact/Priority Score of 30 (10\%)}
	\parbox{6.5in}{\textit{ATIS: Adaptive Time and Item Sampling for Improving Momentary Suicide Risk Identification.}}
	\parbox{6.5in}{Total Costs: \$7,958}
\end{center}
	

% Indiana CTSI COVID-19 Support. Co-Investigator. Coping with Pandemic Distress: Intentional COVID-19 Exposure as a Maladaptive Coping Strategy. Submitted 5/2020. Total Costs: $13722.47.


	\item{\textbf{\large{Completed}}}
	
		\begin{center}
		\parbox{6.5in}{{Ross Jacobucci (PI) }   \hspace{9cm} 7/1/19 - 6/30/21}
		\parbox{6.5in}{Loan Repayment Programs (Clinical)}
		\parbox{6.5in}{\textbf{National Institutes of Mental Health} }
		\parbox{6.5in}{\textit{Integrative Data Analysis for Increasing the Prediction of Suicide}}
	\end{center}
	
		\begin{center}
		\parbox{6.5in}{{Ross Jacobucci (PI)}   \hspace{9.2cm} 1/1/20-12/31/20}
		\parbox{6.5in}{\textbf{Notre Dame Faculty Initiation Grant.} }
		\parbox{6.5in}{\textit{Dynamic Multimodal Assessment of Suicide Risk: An Initial Application.}}
		\parbox{6.5in}{Total Direct: \$9,930}
	\end{center}
	
		\begin{center}
		\parbox{6.5in}{{Ross Jacobucci (PI) }   \hspace{9cm} Academic Year 2018-2019}
		\parbox{6.5in}{Small Research and Creative Work grant}
		\parbox{6.5in}{\textbf{Institute for Scholarship in the Liberal Arts (ND)} }
		\parbox{6.5in}{\textit{Text Analysis for the Prediction of Suicide: A Pilot Study}}
		\parbox{6.5in}{\$2,475}
	\end{center}
	
	\begin{center}
		\parbox{6.5in}{{Ross Jacobucci (Co-PI) \& Brooke Ammerman (Co-PI)}   \hspace{3.8cm} 8/13/18 - 8/13/20}
		\parbox{6.5in}{\textbf{Advanced Diagnostics and Therapeutics Discovery Fund.} }
		\parbox{6.5in}{\textit{Using Integrative Data Mining to Improve the Prediction of Suicide: An Initial Application.}}
		\parbox{6.5in}{Total Direct: \$41,068}
	\end{center}

	
\end{itemize}


%%%%%%%%%%%%%%%%%%%%%%%%%%%%%%
\resheading{Awards/Honors}
%%%%%%%%%%%%%%%%%%%%%%%%%%%%%%
%\begin{center}
%	\parbox{6.5in}{\textbf{Ruth L. Kirschstein National Research Service Award}}
%	\parbox{6.5in}{Predoctoral trainee on the Multidisciplinary Research Training in Gerontology Grant at USC.} 
%	\parbox{6.5in}{Three years of full funding.}
%\end{center}
\begin{center}

	\parbox{6.5in}{\textbf{2021 Elected Member}}
	\vspace{0.05in}
	\parbox{6.5in}{Society of Multivariate Experimental Psychology.}
	
	\parbox{6.5in}{\textbf{2021 APS Rising Star}}
	\vspace{0.05in}
	\parbox{6.5in}{Association for Psychological Science.}
	
	\parbox{6.5in}{\textbf{2019 Anastasi Dissertation Award}}
	\vspace{0.05in}
	\parbox{6.5in}{Division 5, American Psychological Association}
	%\parbox{6.5in}{Amount: \$500}
	
\end{center}

\begin{comment}
	\parbox{6.5in}{\textbf{Academic All-American}, Golf, 2009, 2010}
\parbox{6.5in}{An award given out by the Golf Coaches Association of America based on academic and athletic performance.}
\begin{center}
\parbox{6.5in}{\textbf{2016 USC Psychology Department Travel Grant Award}}
\vspace{0.05in}
\parbox{6.5in}{Travel award to present at the 2016 Psychometric Society Meeting.}
\parbox{6.5in}{Amount: \$500}
\end{center}
\begin{center}
\parbox{6.5in}{\textbf{2015 SMEP Conference Travel Award}}
\vspace{0.05in}
\parbox{6.5in}{Travel award to present at the 13th Annual Society of Multivariate Experimental Psychology Graduate Student Conference.}
\parbox{6.5in}{Amount: \$1500}
\end{center}
\begin{center}
\parbox{6.5in}{\textbf{2015 ISIR Conference Travel Award}}
\parbox{6.5in}{Travel award to present at the annual meeting of the International Society for Intelligence Research.}
\parbox{6.5in}{Amount: \$1500}
\end{center}
\begin{center}
\parbox{6.5in}{\textbf{2014 SMEP Workshop Travel Award}}
\parbox{6.5in}{Travel award to attend the 2014 APA Advanced Training Institutes on Structural Equation Modeling in Longitudinal Research and Exploratory Data Mining in Behavioral Research. }
\parbox{6.5in}{Amount: \$1000}
\end{center}
\begin{center}
\parbox{6.5in}{\textbf{2014 APA Science Directorate Travel Award}}
\parbox{6.5in}{Travel award to attend the 2014 APA Advanced Training Institutes on Structural Equation Modeling in Longitudinal Research and Exploratory Data Mining in Behavioral Research.}
\parbox{6.5in}{Amount: \$500}
\end{center}
\begin{center}
\parbox{6.5in}{\textbf{Intercollegiate Academics Fund Travel Award}, UNI, Fall 2012}
\parbox{6.5in}{Award allocated to promote and support the presentation of research at an academic conference.}
\parbox{6.5in}{Amount: \$650}
\end{center}
\begin{center}
\parbox{6.5in}{\textbf{Graduate Student Opportunity Fund For Travel}, UNI, Fall 2012}
\parbox{6.5in}{Award to support travel for graduate students presenting at an academic conference.}
\parbox{6.5in}{Amount: \$250}
\end{center}
\begin{center}
\parbox{6.5in}{\textbf{College of Social and Behavioral Sciences Graduate Research Award}, UNI, Fall 2012}
\parbox{6.5in}{Award allocated for the purchase of expenses related to thesis research.}
\parbox{6.5in}{Amount: \$500}
\end{center}
\begin{center}
\parbox{6.5in}{\textbf{Graduate College Tuition Scholarship}, UNI,  Fall 2012}
\parbox{6.5in}{Scholarship based on academic performance.}
\end{center}
\end{comment}



%%%%%%%%%%%%%%%%%%%%%%%%%%%%%%
\resheading{Books}
%%%%%%%%%%%%%%%%%%%%%%%%%%%%%%
\begin{itemize} 
	\setlength{\topsep}{0pt}%
	\setlength{\leftmargin}{0.1in}%
	\setlength{\listparindent}{-0.1in}%
	\setlength{\itemindent}{-0.2in}%
	\setlength{\parsep}{\parskip}%
	
	\item[]\textbf{Jacobucci, R.,} Grimm, K. J., \& Zhang, Z. (forthcoming). \emph{Machine learning for social and behavioral research.} New York, NY: Guilford.
\end{itemize}


%%%%%%%%%%%%%%%%%%%%%%%%%%%%%%
\resheading{Publications}
%%%%%%%%%%%%%%%%%%%%%%%%%%%%%%

* refers to undergraduate students, @ graduate advisees ($@_{1}$ for graduate advisees that I was not the primary mentor), and (+) for postdocs.
\begin{itemize} 
	\setlength{\topsep}{0pt}%
	\setlength{\leftmargin}{0.1in}%
	\setlength{\listparindent}{-0.1in}%
	\setlength{\itemindent}{-0.2in}%
	\setlength{\parsep}{\parskip}%
	
	\item {\textbf{\large{Under Review Preprints}}}
\end{itemize}
\begin{etaremune}


\item $@$McClure, K., \& \textbf{Jacobucci, R.} (2021, August 6). Are Items More Than Indicators? An examination of psychometric homogeneity, item-specific effects, and consequences for structural equation models. https://doi.org/10.31234/osf.io/n4mxv



	
	\end{etaremune}
\vspace{3mm}
\begin{itemize}
	%
	\item {\textbf{\large{Peer Reviewed Publications}}}
\end{itemize}
\begin{etaremune}
	
	
	\item \textbf{Jacobucci, R.,} $@$McClure, K.,  \& Ammerman, B. (in press). Comparing the Role of Perceived Burdensomeness and Thwarted Belongingness in Prospectively Predicting Active Suicidal Ideation. \emph{Suicide and Life-Threatening Behaviors}.
	% 
	
	\item $@$Wilcox, K. T., \textbf{Jacobucci, R.}, Zhang, Z., \& Ammerman, B. A. (in press). Supervised Latent Dirichlet Allocation with Covariates: A Bayesian Structural and Measurement Model of Text and Covariates. \emph{Psychological Methods}.
	
	
	\item Case, J.A.C., Sullivan-Toole, H., Mattoni, M., \textbf{Jacobucci, R.}, Forbes, E.E., \& Olino, T.M. (2022). Evaluating the Item-Level Factor Structure of Anhedonia. \emph{Journal of Affective Disorders, 299}, 215-222.
	
	\item Grimm, K. J., \textbf{Jacobucci, R.,} Stegmann, G., \& Serang, S. (2022). Explorations of individual change processes and their determinants: A novel approach and remaining challenges. \emph{Multivariate Behavioral Research, 57}, 525-542.
	%
		\item \textbf{Jacobucci, R.} (2022). A Critique of Using the Labels Confirmatory and Exploratory in Modern Psychological Research. \textit{Frontiers in Psychology, 13}. https://doi.org/10.3389/fpsyg.2022.1020770
	%
	\item $@$Li, X., \& \textbf{Jacobucci, R.} (2022). Regularized structural equation modeling with stability selection. \emph{Psychological Methods, 27}, 497–518. https://doi.org/10.1037/met0000389
	%
		\item \textbf{Jacobucci, R.} \& $@$Li, X. (2022). Does Minority Case Sampling Improve Performance with Imbalanced Outcomes in Psychological Research? \emph{Journal of Behavioral Data Science, 2(1)}, 1–16. DOI:https://doi.org/10.35566/jbds/v2n1/p3
	%
	\item Burke, T. A., $@$Shao, S., \textbf{Jacobucci, R.,} Kautz, M., Alloy, L. B., \& Ammerman, B. A. (2022). Examining momentary associations between behavioral approach system indices and nonsuicidal self-injury urges. \emph{Journal of Affective Disorders, 296.} 244-249.
	%
		\item Ammerman, B.A., Burke, T. A., \textbf{Jacobucci, R.}, $@$McClure, K., \& Lui. R. T. (2021). A prospective examination of COVID-19-related social distancing practices on suicidal ideation. \emph{Suicide and Life-Threatening Behavior, 51}, 969-977.
	%
	\item Ammerman, B. A., Burke, T. A., \textbf{Jacobucci, R.}, \& $@$McClure, K. (2021). How we ask matters: The impact of question wording in single-item measurement of suicidal thoughts and behaviors. \emph{Preventive Medicine, 152}, 106472.
	%
	\item Ortiz, S. N., Forrest, L. N., Kinkel-Ram, S. S., \textbf{Jacobucci, R. C.}, \& Smith, A. R. (2021). Using shape and weight overvaluation to empirically differentiate severity of other specified feeding or eating disorder. \emph{Journal of Affective Disorders, 295}, 446-452.
	%
	\item $@$Li, X. \& \textbf{Jacobucci, R.,} \& Ammerman, B. A. (2021). Tutorial on the use of the regsem package in R. \emph{Psych, 3}. 579-593.
	%
	%
	%
		\item Ammerman, B.A., Burke, T.A., \textbf{Jacobucci, R.}., \& $@$McClure, K. (2021). Preliminary investigation of the association between COVID-19 and suicidal thoughts and behaviors in the U.S. \emph{Journal of Psychiatric Research, 134}, 32-38.
	%
	\item $+$Ober, T., Cheng, Y., \textbf{Jacobucci, R.,}, \& Whitney, B, M. (2021). Examining the within-domain factor structure of the big five inventory-2 with an adolescent sample. \emph{Psychological Assessment, 33}, 14-28.
	%
		\item Rodgers, D. M., \textbf{Jacobucci, R.,} \& Grimm, K. J. (2021). A multiple imputation approach for handling missing data in classification and regression trees. \emph{Journal of Behavioral Data Science, 1}, 127-153..
	%
	\item Littlefield, A. K., Cooke, J. T., Bagge, C., Glenn, C., Kleiman, E. M., \textbf{Jacobucci, R.}, Millner, A. J., \& Steinley, D. (2021). Machine Learning to Classify Suicidal Thoughts and Behaviors: Implementation within the Common Data Elements used by the Military Suicide Research Consortium. \emph{Clinical Psychological Science, 9}, 467-481.
	%
	\item \textbf{Jacobucci, R.,} Littlefield, A., Millner, A. J., Kleiman, E. M., \& Steinley, D. (2021). Evidence of inflated prediction performance: A commentary on machine learning and suicide research. \emph{Clinical Psychological Science, 9}, 129-134.
	% 
	\item \textbf{Jacobucci, R.,} Ammerman, B. A., \& $@$Wilcox, K. (2021). The application of machine learning for text-based responses to improve suicide risk prediction. \emph{Suicide and Life Threatening Behavior, 51}, 55-64.
	%
	\item \textbf{Jacobucci, R.,} Ammerman, B. A., \& $@$Li, X. (2021). Using ordinal regression for advancing the understanding of distinct suicide outcomes. \emph{Suicide and Life Threatening Behavior, 51}, 65-75.
	%
	\item Grimm, K. J., \& \textbf{Jacobucci, R.} (2021). Reliable trees: Reliability informed recursive partitioning for psychological data. \emph{Multivariate Behavioral Research, 56,} 595-607.
	%
	\item  Serang, S., \textbf{Jacobucci, R.}, Stegmann, G., Brandmaier, A. M., Culianos, D., \& Grimm, K. J. (2021). Mplus Trees: Structural equation model trees using Mplus. \emph{Structural Equation Modeling, 28}, 127-137.
	%
		\item Forrest, L., \textbf{Jacobucci, R.,} \& Grilo, C. M. (2020). Empirically-Determined severity levels for binge-eating disorder outperform existing severity classification schemes. \emph{Psychological Medicine, 1-11}.
	%
		\item Jiang, M., Ammerman, B.A., $@_1$Zeng, Q., \textbf{Jacobucci, R.}, \& $@_1$Brodersen, A. (2020). Phrase-level pairwise topic modeling to uncover helpful peer responses to online suicidal crises. \emph{Humanities \& Social Sciences Communications, 7}, 1-13.
	%
	\item $@$Nelson, N. A., \textbf{Jacobucci, R.,} Grimm, K. J., \& Zelinski, E. (2020). The Bi-directional relationship between physical health and memory. \emph{Psychology \& Aging, 35}, 1140-1153.
	%
	\item  $@_1$Hong, M., \textbf{Jacobucci, R.}, \& Lubke, G. (2020). Deductive Data Mining. \emph{Psychological Methods, 25}, 691-707.
	%
	%
	\item  Ammerman, B.A., \textbf{Jacobucci, R.}, Turner, B. J., Dixon-Gordon, K., \& McCloskey, M.S. (2020). Quantifying the importance of lifetime frequency versus number of methods used in the consideration of NSSI severity. \emph{Psychology of Violence, 10} 442-451.
	%
	\item Liang, X., \& \textbf{Jacobucci, R.} (2020). Regularized Structural Equation Modeling to Detect Measurement Bias: Evaluation of Lasso, Adaptive Lasso and Elastic Net. \emph{Structural Equation Modeling, 27}, 722-734.
	%
	\item Vargas, I., Haeffel, G.J., \textbf{Jacobucci, R.}, Boyle, J.T., Mayer, S.E., \& Lopez-Duran, L. (2020). Negative Cognitive Style And Cortisol Reactivity To A Laboratory Stressor: A Preliminary Study. \emph{International Journal of Cognitive Therapy, 13}, 1-14. 
	%
	\item Burke, T. A., \textbf{Jacobucci, R.}, Ammerman, B. A., \& Diamond, G. (2020). Using machine learning to classify suicide attempt history among youth in medical care settings. \emph{Journal of Affective Disorders, 268}, 206-214.
	%
	\item  \textbf{Jacobucci, R.,} \& Grimm, K. J. (2020). Machine learning and psychological research: The unexplored effect of measurement. \emph{Perspectives on Psychological Science, 15}, 809-816.
	% 
	\item Serang, S. \& \textbf{Jacobucci, R.} (2020). Exploratory mediation analysis of dichotomous outcomes via regularization. \emph{Multivariate Behavioral Research, 55}, 69-86.
	%
	\item  $@_1$Nelson, N. A., Bergeman, C. S., \& \textbf{Jacobucci, R.} (2019). Future time perspective in mid-to-later life: The role of personality. \emph{The Journal of Gerontology: Psychological Science}, gbz110.
	%
	\item  Ammerman, B. A., Hong, M., Sorgi, K. M., \textbf{Jacobucci, R.}, Park, Y., \& McCloskey, M. S. (2019). An examination of individual forms of nonsuicidal self-injury. \emph{Psychiatry Research, 278,} 268-274.
	%
	\item  Burke, T.A., Ammerman, B.A., \& \textbf{Jacobucci, R.} (2019). The use of machine learning in the study of suicidal and non-suicidal self-injurious thoughts and behavior: A systematic review. \emph{Journal of Affective Disorders, 245,} 869-884.
	%
	\item  \textbf{Jacobucci, R.}, Serang, S., \& Grimm, K. J. (2019). A short note on complications in interpretation with the dual change score model . \emph{Structural Equation Modeling, 26}, 924-930.
	%
	\item  Ammerman, B.A., \textbf{Jacobucci, R.} \& McCloskey, M.S. (2019). Re-considering important outcomes of the NSSI disorder diagnostic criterion A. \emph{Journal of Clinical Psychology, 75}, 1084-1097.
	% 
	\item  \textbf{Jacobucci, R.}, Brandmaier, A., \& Kievit, R. (2019). A practical guide to variable selection in structural equation models with regularized MIMIC models. \emph{Advances in Methods and Practices in Psychological Science, 2}, 55-76.  
	%
	\item  Usami, S., \textbf{Jacobucci, R.}, \& Hayes, T. (2019). The performance of latent growth curve model based structural equation model trees to uncover population heterogeneity in growth trajectories. \emph{Computational Statistics, 34}, 1-22.
	%
	\item Ammerman, B. A., Serang, S., \textbf{Jacobucci}, R., Burke, T., A., Alloy, L. B., \& McCloskey, M. S. (2018). Exploratory analysis of mediators in the relationship between childhood maltreatment and suicidal behavior. \emph{Journal of Adolescence, 69, 103-112.}
	%
	\item Stegmann, G., \textbf{Jacobucci, R.}, Serang, S., \& Grimm, K. J. (2018). Recursive partitioning with nonlinear change trajectories. \emph{Multivariate Behavioral Research, 53}, 559-570.
	%
	\item Burke, T. A., \textbf{Jacobucci, R.}, Ammerman, B. A.,, Piccirillo, M., McCloskey M., \& Alloy, L. B. (2018). Identifying the relative importance of non-suicidal self-injury features in predicting suicidal ideation and behavior using exploratory data mining. \emph{Psychiatry Research, 262,} 175-183.
	%
	\item Ammerman, B. A., \textbf{Jacobucci, R.}, \& McCloskey, M. S. (2018). Using exploratory data mining to identify important predictors of non-suicidal self-injury frequency. \emph{Psychology of Violence, 8,} 515-525.
	%
	\item Ammerman, B. A., \textbf{Jacobucci, R.}, Kleiman, E. M., Uyeji, L., \& McCloskey, M. S. (2018). The relationship between nonsuicidal self-injury age of onset and severity of self-harm. \emph{Suicide and Life Threatening Behavior, 48,} 31-37.
	%
	\item  \textbf{Jacobucci, R.}, Grimm, K. J. (2018). Comparison of frequentist and Bayesian regularization in structural equation modeling. \emph{Structural Equation Modeling, 25}, 639-649.
	%
	\item Stegmann, G., \textbf{Jacobucci, R.}, Harring, J., \& Grimm, K. J. (2018). Nonlinear mixed-effects modeling programs in R. \emph{Structural Equation Modeling, 25}, 160-165.
	%
	\item Serang, S., \textbf{Jacobucci, R.}, Brimhall, K. C., \& Grimm, K. J. (2017). Exploratory mediation analysis via regularization. \emph{Structural Equation Modeling, 24}. 733-744.
	%
	\item \textbf{Jacobucci, R.}, Grimm, K. J., \& McArdle, J. J. (2017). A comparison of methods for uncovering sample heterogeneity: Structural equation model trees and finite mixture models. \emph{Structural Equation Modeling, 24}. 270-282.
	%
	\item Ammerman, B. A., \textbf{Jacobucci, R.,}, Kleiman, E. M., Muehlenkamp, J. J., \& McCloskey, M. S. (2017). Development and validation of empirically derived frequency criteria for NSSI disorder using exploratory data mining, \emph{Psychological Assessment, 29,}221-231.
	%
	\item \textbf{Jacobucci, R.}, Grimm, K. J., \& McArdle, J. J. (2016). Regularized structural equation modeling, \emph{Structural Equation Modeling, 23}, 555-566. doi:10.1080/10705511.2016.1154793. PMCID: 4937830
	%
	%
	%\item \textbf{Jacobucci, R.,} \& McArdle, J. J. (2015). Abstract: Regularized structural equation modeling, \emph{Multivariate Behavioral Research 50}, 736-736. doi:10.1080/00273171.2015.1121125. 
	%
	\item Hayes, T., Usami, S., \textbf{Jacobucci, R.,} \& McArdle, J. J. (2015). Using classification and regression trees (CART) and random forests to analyze attrition in longitudinal data: Results from two simulation studies, \emph{Psychology and Aging, 30}, 911-929. doi:10.1037/pag0000046. PMCID: 4743660
	%
	\item Skaar, N. R., Christ, T. J., \& \textbf{Jacobucci, R.} (2014). Measuring adolescent prosocial and health risk behavior in schools: Initial development of a screening measure. \emph{School Mental Health, 6}, 137-149. doi:10.1007/s12310-014-9123-y
	
\end{etaremune}
%
\vspace{3mm}
\begin{itemize}
	%
	\item {\textbf{\large{Other Publications}}}
\end{itemize}
\begin{etaremune}
	\item  \textbf{Jacobucci, R.,} $@$Shao, S., \& $@$Hattori, M. (2020). On the integration of machine learning and psychometrics. \emph{The Score (APA Div. 5 Newsletter)}.
	%
	\item  $@$Hong, M. R., \textbf{Jacobucci, R.}. (2019). Review of Growth Modeling: Structural Equation and Multilevel Modeling Approaches (Grimm, Ram \& Estabrook, 2017). \emph{Psychometrika, 84}, 327-332.
	%
	\item Grimm, K. J., \textbf{Jacobucci, R.}, McArdle, J. J. (January, 2017). Big data methods and psychological science. \emph{Psycholgocical Science Agenda}.
	
\end{etaremune}
\vspace{3mm}
%
%
\begin{itemize}
	\item {\textbf{\large{Chapters}}}
\end{itemize}
\begin{etaremune}
	\item Brandmaier, A. M. \& \textbf{Jacobucci, R.} (forthcoming). Machine-learning approaches to structural
	equation modeling. In R. H. Hoyle (Ed.), Handbook of structural equation modeling.
	Guilford Press.
	%
	\item Grimm, K. J., Stegmann, G., \textbf{Jacobucci, R.}, \& Serang, S. (2020). Big data in developmental psychology. In S. E. Woo, L. Tay, \& R. W. Proctor (Eds.), Big data methods for psychological research: New horizons and challenges. Washington, DC: American Psychological Association.
	%
	\item  \textbf{Jacobucci, R.} \& Grimm, K. J. (2018). Regularized estimation of multivariate latent change score models. Advances in Longitudinal Models for Multivariate Psychology: A Festschrift for Jack McArdle, 109-125. Routledge, London.
	%
	\item Grimm, K. J. \& \textbf{Jacobucci, R.} (2018). Individually varying time metrics in latent change score models. Advances in Longitudinal Models for Multivariate Psychology: A Festschrift for Jack McArdle. Routledge, London.
	%
	%
\end{etaremune}
\begin{itemize}
	\vspace{3mm}
	%
	%\newpage
	\item {\textbf{\large{Technical Reports}}}
\end{itemize}
\begin{etaremune}
	\item \textbf{Jacobucci, R.}  (2018, January 18). Decision tree stability and its effect on interpretation. \emph{PsyArXiv preprint: https://doi.org/10.31234/osf.io/f2utw}.
	%
	\item \textbf{Jacobucci, R.} (2017). regsem: Regularized Structural Equation Modeling. \emph{arXiv preprint arXiv:1703.08489}.
	%
	\vspace{3mm}
	%
	%
\end{etaremune}


%\begin{itemize}
%	\vspace{3mm}
	%
	%\newpage
%	\item {\textbf{\large{In Popular Press}}}
%\end{itemize}
%\begin{etaremune}
%	\item https://www.psychologicalscience.org/observer/machine-learning-transforming-psychological-science

%	\vspace{3mm}
	%
	%
%\end{etaremune}


%%%%%%%%%%%%%%%%%%%%%%%%%%%%%%
\resheading{Presentations}
%%%%%%%%%%%%%%%%%%%%%%%%%%%%%%
\begin{itemize} 
	\setlength{\topsep}{0pt}%
	\setlength{\leftmargin}{0.1in}%
	\setlength{\listparindent}{-0.1in}%
	\setlength{\itemindent}{-0.2in}%
	\setlength{\parsep}{\parskip}%
	
	
	
	%\item {\textbf{\large{Symposium Chair}}}
	
	%\item[] \emph{Structural Equation Modeling}. 2016 International Meeting for the Psychometric Society, Asheville, North Carolina.
	\item {\textbf{\large{Invited Presentations}}}
	%
	\item[]\textbf{Jacobucci, R.} (2022, November). \emph{Flexible Specification of Large Structural Equation Models with Regularization.} Invited talk given at the University of Illinois Quant Brown Bag Speaker Series.
	%
	\item[]\textbf{Jacobucci, R.} (2022, May). \emph{Heterogeneity in the Pathways to Suicidal Ideation.} Invited talk given at the 34th Annual Convention for Association for Psychological Science, Chicago, IL.
	%
	\item[]\textbf{Jacobucci, R.} (2022, May). \emph{Heterogeneity in Suicide.} Invited talk given at the University of Virginia's Psychology Colloquium.
	%
	\item[]\textbf{Jacobucci, R.} (2022, February). \emph{Measurement and Prediction.} Invited talk given at the UC Davis Quantitative Psychology Brown Bag Speaker Series.
	%
	\item[]\textbf{Jacobucci, R.} (2020, October). \emph{Theory based dynamic text analysis.} Invited talk given to the Screenomics Lab at Stanford.
	%
	\item[]\textbf{Jacobucci, R.} (2020, October). \emph{Using imperfectly measured predictors in machine learning.} Workforce Science workshop at Rice University in Houston, Texas. (Conference canceled)
	%
	\item[]\textbf{Jacobucci, R.} (2020, October). \emph{Machine learning for suicide research.} Invited talk given at the UCLA Quantitative Psychology Brown Bag.
	%
	\item[]\textbf{Jacobucci, R.} (2019, May). \emph{Flexible specification of large structural equation models with regularization.} Keynote speech given at the Big Data in Psychology Pre-Conference in Dubrovnik, Croatia. Video: https://www.psycharchives.org/handle/20.500.12034/2118
	%
	\item[]\textbf{Jacobucci, R.} (2019, February). \emph{Novel assessment strategies for improving our understanding of mental health.} Invited talk given at the Quant Methods colloquium at Vanderbilt University.
	%
	\item[]\textbf{Jacobucci, R.,} Hong, M., Brandmaier, A., Ammerman, B., \& McCloskey, M. (2018, October). \emph{Assessing Continuous Versus Categorical Diagnosis Using Latent Variable Modeling.} Invited talk given at the Arizona State Quantative Psychology brown bag.
	%
	\item[]\textbf{Jacobucci, R.,} Hong, M., Brandmaier, A., Ammerman, B., \& McCloskey, M. (2018, October). \emph{Assessing Continuous Versus Categorical Diagnosis Using Latent Variable Modeling.} Invited talk given at the Quantitative Developmental Systems Methodology Core at Penn State University.
	%
	\item[]\textbf{Jacobucci, R.} (2018, July). \emph{Flexible specification of large structural equation models with regularization.} Invited talk given at the Max Planck Institute for Human Development in Berlin, Germany.
	
	%
	\item {\textbf{\large{Oral Presentations}}}
	%
	\item[] \textbf{Jacobucci, R.} (2023, June). Examining the dynamic relationship between alcohol use and NSSI at the group and individual levels. Paper to be presented at the 18th Annual Convention for the International Society for the Study of Self-Injury, Vienna, Austria.
	%
	%
	\item[] $@$Shao, S., \textbf{Jacobucci, R.}, (2021, July). Dynamic Poisson Factor Analysis: A Hierarchical Bayesian Approach with Intensive Text Data. Invited symposium at the International Meeting of the Psychometric Society. Won the best student presentation award sponsored by Duolingo.
	%
	\item[] $@$McClure, K.E., \textbf{Jacobucci, R.}, \& Ammerman, B.A. (2020, March). Disclosure of Suicidality and Open-Response Questions: A Blended Text Mining Approach. Paper submitted for presentation at the 54th Annual Convention for the Association for Behavioral and Cognitive Therapies, Philadelphia, Pennsylvania.
	%
	\item[] \textbf{Jacobucci, R.} (2019, November). The use text mining for clinical research: An overview of the methodology. Paper presented at the 53rd Annual Convention for the Association for Behavioral and Cognitive Therapies, Atlanta, Georgia.
	%
	\item[] Ammerman, B.A., Wilcox, K. T., \& \textbf{Jacobucci, R.} (2019,November) A nuanced examination of the relationship between social connectedness and suicidal ideation. Paper presented at the 53rd Annual Convention for the Association for Behavioral and Cognitive Therapies, Atlanta, Georgia.
	
	\item[] Park, Y., O'Loughlin, C. M., \textbf{Jacobucci, R.} , \& Ammerman, B.A. (2019, November). Examining person-specific reasons for living in the prediction of suicidal ideation: A text mining approach. Paper presented at the 53rd Annual Convention for the Association for Behavioral and Cognitive Therapies, Atlanta, Georgia.
	
	\item[] Wilcox, K. T., \textbf{Jacobucci, R.} , McCloskey, M. S. \& Ammerman, B.A.(2019, November). Evaluating personal narratives of interpersonal relationships via text mining to predict nonsuicidal self-injury. Paper presented at the 53rd Annual Convention for the Association for Behavioral and Cognitive Therapies, Atlanta, Georgia.
	%
	%
	\item[]\textbf{Jacobucci, R.,} Hong, M., Brandmaier, A., Ammerman, B., \& McCloskey, M. (2018, September). \emph{Assessing Continuous Versus Categorical Diagnosis Using Latent Variable Modeling.} Talk given at the Quantitative Studies Group at the University of Notre Dame.
	%
	\item[]Stegmann, G., \textbf{Jacobucci, R.}, Serang, S., \& Grimm, K. J. (2018, July). \emph{Identifying divergent nonlinear growth trajectories using recursive partitioning.} International Meeting of Psychometric Society. New York, NY.
	%
	\item[]Grimm, K. J., Stegmann, G., \textbf{Jacobucci, R.}, \& Serang, S. (2018, March). Machine learning approaches to understanding differences in longitudinal change trajectories. Keynote address at the 6th Annual Meeting of the Texas Universities' Educational Statistics and Psychometrics Alliance.
	%
	\item[]\textbf{Jacobucci, R.} (2017, November). \emph{Exploratory data mining for a single outcome in clinical research.} Paper presented at the 51st Annual Convention for the Association for Behavioral and Cognitive Therapies, San Diego, California.
	%
	\item[]Ammerman, B. A., \textbf{Jacobucci, R.}, Turner, B. J., Serang, S., \& McCloskey, M. S. (2017, November). \emph{Quantifying the importance of lifetime frequency versus number of methods used in the consideration of NSSI severity.} Paper presented at the 51st Annual Convention for the Association for Behavioral and Cognitive Therapies, San Diego, California.
	%
	\item[]\textbf{Jacobucci, R.}, Serang, S., \& Ammerman, B.A. (2017, November). \emph{Multivariate approaches to exploratory data mining with the application of exploratory mediation for self-injury.} Paper presented at the 51st Annual Convention for the Association for Behavioral and Cognitive Therapies, San Diego, California.
	%
	\item[]Burke, T. A., \textbf{Jacobucci, R.}, Ammerman, B.A., Hamilton, J. L., \& Alloy, L. B. (2017, June). \emph{Using exploratory data mining to compare importance of risk factors in predicting recent versus former history of non-suicidal self-injury.} Paper presented at the 51st Annual Convention for the Association for Behavioral and Cognitive Therapies, San Diego, California.
	%
	\item[] Kleiman, E.M., \textbf{Jacobucci, R.}, Ammerman, B.A., Turner, B.J., Beale, E.E., Fedeor, S., Huffman, J.C., \& Nock, M. K. (2017, June). \emph{What affect states are most strongly associated with suicidal ideation? A real-time monitoring, exploratory data mining study.} Paper presented at the 51st Annual Convention for the Association for Behavioral and Cognitive Therapies, San Diego, California.
	%
	
	%
	\item[] Zelinski, E. M., \textbf{Jacobucci, R.} (2017, July). \emph{Heterogeneity in cognitive change trajectories of older adults observed from structural equation model trees.} Alzheimer’s Association International Conference in London, UK
	%
	\item[]Burke, T. A., \textbf{Jacobucci, R.}, Ammerman, B. A.,, Piccirillo, M., McCloskey M., \& Alloy, L. B. (2017, April). \textit{Identifying the relative importance of non-suicidal self-injury features in predicting suicidal ideation and behavior using exploratory data mining.} Paper presented at the 50th Annual Conference of American Association of Suicidology in Pheonix, Arizona.
	
	\item[] \textbf{Jacobucci, R.}, Zelinski, E. M. (2016, November). \emph{The bi-directional relationship between health and cognition in older age}. Paper presented at the Gerontological Society of America Annual Meeting, New Orleans, Louisiana.
	
	\item[] Peters, K., \textbf{Jacobucci, R.}, Prescott, C., Walters, E., McArdle, J. J. (2016, November). \emph{Computerized adaptive testing of cognitive abilities in the Project Talent Aging Study}. Paper presented at the Gerontological Society of America Annual Meeting, New Orleans, Louisiana.
	
	\item[] \textbf{Jacobucci, R.}, Grimm, K. J. (2016, October). \emph{Psuedo-continuous testing of the latent difference score model}. Paper invited for presentation at the conference to honor the work of Jack McArdle, Richmond, Virginia.
	%
	\item[] Grimm, K. J., \textbf{Jacobucci, R.} (2016, October). \emph{ Individually-varying time metrics in latent change score models}. Paper invited for presentation at the conference to honor the work of Jack McArdle, Richmond, Virginia.
	%
	%\item[] \textbf{Jacobucci, R.}, Zelinski, E. M. (2016, July). \emph{Exploratory search for heterogeneity in change across older age using structural equation model trees.}. Abstract submitted for paper presentation at the 21st IAGG World Congress of Gerontology and Geriatrics, San Francisco, California.
	
	\item[] \textbf{Jacobucci, R.}, Grimm, K. J., McArdle, J. J. (2016, July). \emph{Comparison of frequentist and Bayesian regularization in structural equation modeling.} Paper presented at the International Meeting for the Psychometric Society, Asheville, North Carolina.
	%
	\item[]Ammerman, B. A., \textbf{Jacobucci, R.,} Kleiman, E. M., Muehlenkamp, J., J. \& McCloskey, M. S. (2015, November). \emph{The exploration and validation of an empirically derived frequency criteria for NSSI Disorder.} Paper presented at the 49th Annual Convention for the Association for Behavioral and Cognitive Therapies, Chicago, Illinois
	%
	\item[] \textbf{Jacobucci, R.}, Prindle, J. J., McArdle, J. J. (2015, September). \emph{An Application of Exploratory Data Mining in Project TALENT.} Paper presented at the International Meeting for Intelligence Research Annual Conference, Albuquerque, New Mexico.
	%
	\item {\textbf{\large{Poster Presentations}}}
	
	\item[] \textbf{Jacobucci, R.}, \& Zelinski, E. M. (2017, July). \emph{Exploratory Search for Heterogeneity in Change Across Old Age Using Structural Equation Model Trees.} Poster presented at the International Association of Gerontology and Geriatrics, San Francisco, California.
	
	\item[]Khoddam, R., Cho, J., Jacobucci, R., Prescott, C. A., \& Leventhal, A. M. (2017, March). \textit{Bivariate Latent Difference Score Modeling to Examine the Relationship between Conduct Problems and Cigarette Smoking among Adolescents.} Poster presented at the Society for Research on Nicotine and Tobacco annual meeting, Florence, Italy.    
	
	\item[]Ammerman, B. A., \textbf{Jacobucci, R.}, \& McCloskey, M. S. (2016, June). \textit{The Prediction of Non-Suicidal Self-Injury Frequency Using Random Forests}. Poster presented at the 11th Annual Meeting of the International Society for the Study of Self-Injury, Eau Claire, Wisconsin.
	%
	\item[] \textbf{Jacobucci, R.}, Hayes, T., Zelinski, E. M. (2015, November). \emph{Evaluating Methods for Generalizing from a Convenience Sample.} Poster presented at the 68th Annual Scientific Meeting of the Gerontological Society of America, Orlando, Florida.
	%
	\item[]Zelinski, E., \textbf{Jacobucci, R.}, Kennison, R., \& Zak, D. (2014, November). \emph{Can a convenience sample produce generalizable results?} Poster session presented at the Gerontological Society of America Annual Scientific Meeting, Washington, DC.
	%
	\item[]\textbf{Jacobucci, R.}, Williams, J. E., \& Thiruselvam, I. (2013, January). \emph{A confirmatory factor analysis of the short form for the IPIP-NEO five-factor model personality scale.} Poster session presented at the Society for Personality and Social Psychology, New Orleans, Louisiana.
	%
	\item[]\textbf{Jacobucci, R.}, Brownfield, C., \& Williams, J. E. (2012, December). \emph{Intelligence as a factor in criminal offender risk assessment.} Poster session presented at the International Society for Intelligence Research, San Antonio, Texas.
	%
	\item []\textbf{Jacobucci, R.}, Williams, J. E., \& Thiruselvam, I. (2012, March). \emph{Convergent validity of two short form measures of the five factor model of personality.} Poster session presented at the Iowa Psychological Association Spring Conference, Ames, Iowa.
	
\end{itemize}


%%%%%%%%%%%%%%%%%%%%%%%%%%%%%%
\resheading{Software}
%%%%%%%%%%%%%%%%%%%%%%%%%%%%%%
\begin{itemize}
	%
	%\item[]{\textbf{\large{Published}}}
	%
	\item[]\textbf{Jacobucci, R.} et al. (2022). regsem: Performs Regularization on Structural Equation Models (version 1.9.3) [Software]. Available from https://cran.r-project.org/web/packages/index.html\\
-- Downloads as of 01/23: $>$ 300,000
%
\item[]Serang, S., \textbf{Jacobucci, R.}, Grimm, K. J., Stegmann, G., \& Brandmaier, A. M. (20202). MplusTrees: Decision Trees with Structural Equation Models Fit in 'Mplus' (version 0.2.2) [Software]. Available from https://cran.r-project.org/web/packages/MplusTrees/index.html\\
-- Downloads as of 01/23: $>$  16,000
%
\item[]\textbf{Jacobucci, R.}, Stewart, S., Abdolell, M., Serang, S., \& Stegmann, G. (2018). longRPart2: Recursive Partitioning of Longitudinal Data (version 0.2.3) [Software]. Available from https://cran.r-project.org/web/packages/longRPart2/index.html\\
-- Downloads as of 01/23: $>$ 21,000
	%
	\item[]\textbf{Jacobucci, R.} (2018). dtree: Decision Trees (version 0.4.2) [Software]. Available from https://cran.r-project.org/web/packages/dtree/index.html
	%
	\item[]\textbf{Jacobucci, R.} (2016). autosem: Performs Specification Search in Structural Equation Models (version 0.1.0) [Software]. Available from https://cran.r-project.org/web/packages/autoSEM/index.html
	%
\end{itemize}



%\newpage

%%%%%%%%%%%%%%%%%%%%%%%%%%%%%%
%\resheading{Research Experience}
%%%%%%%%%%%%%%%%%%%%%%%%%%%%%%
%\begin{itemize} 
%\setlength{\topsep}{0pt}%
%\setlength{\leftmargin}{0.1in}
%
%\item \textbf{Research Assistant} {Fall 2014-Present }\\
%Supervisor: Dr. Elizabeth Zelinski, USC\\ 
%I am currently using both structural equation models and statistical learning methods to compare the Long Beach Longitudinal Study to the Health and Retirement Study. Some of the main analyses include using the sample weights in HRS to create sample weights in LBLS as well as using SEM to model cognitive decline in LBLS. My main focus is on evaluating different methods of creating sample weights can make the results of a convenience sample, LBLS, more generalizable and to what degree.

%
%\item \textbf{Research Assistant} {Fall 2013-Present }\\
%Supervisor: Dr. John J. McArdle, USC\\ 
%In working with data from Project TALENT (PT), I have conducted both exploratory and confirmatory factor analyses in R and MPlus to determine the factor structure of 60+ cognitive variables. Using these models, in addition to 2000+ other variables collected in the initial and follow up samples, I have used the semtree package to determine relationship between these covariates and factor model parameters. Because of the complexity of models and size of sample, I have utilized the Center for High-Performance Computing and Communication's (HPCC) supercomputer. Finally, I have used multiple item response theory(IRT) models on item level data from 20 cognitive ability scales to determine the best model.
%
%\item \textbf{Psychometric Expert} {Summer 2013-Spring 2014}
%Supervisor: Dr. John J. McArdle, USC\\ 
%Worked closely with Dr. Jiu-Chiuan Chen and Dr. Yong Cen through the Preventative Medicine department at USC in the modeling and prediction of cognitive decline in two large longitudinal datasets. Responsibilities include using structural equation modeling (SEM) to construct models to predict Alzheimer's disease, item response theory and factor analysis to evaluate the factor structure and precision of multiple cognitive measures, and Monte Carlo simulation to conduct multiple power analyses for grant preparation. I have also built on this work in conducting first and second-order latent growth models, mixture models, among other multivariate techniques. 

%
%\item \textbf{Principal Investigator} {Summer 2012-Spring 2013, \emph{Personality and Cognitive Ability in Civilly Committed Sexual Offenders}}\\
%Supervisor: Dr. John E. Williams, UNI\\
%Conducting group administration of cognitive ability and personality measures at the Civil Commitment Unit for Sexual Offenders in Cherokee, Iowa. Responsibilities included conducting a literature review, designing an original study, coordinating with administrators at the civil commitment unit, administration of ability and self-report measures to patients, entering, cleaning, and analyzing data in SPSS.
%
%\item \textbf{Principal Investigator} {Summer 2012-Fall 2012, \emph{Personality, Cognitive Functioning, Adaptive Behavior, and Criminogenic Needs}}\\
%Supervisor: Dr. John E. Williams, UNI\\
%Conducting archival data collection at the Iowa Medical and Classification Center in 	Coralville, Iowa. Examine cognitive ability, academic achievement, personality, risk assessment, and conviction charges. 
%
%\item \textbf{Principal Investigator} {Fall 2011-Summer 2012, \emph{The Role of Grit, Self-Control, and Conscientiousness in Academic Achievement}}\\
%Supervisor: Dr. John E. Williams, UNI\\
%Conducted group administration of self-report measures of personality and academic achievement in a college sample. Responsibilities included the administration of self-report measures to high school and college populations, entering, cleaning, and analyzing data in SPSS, and writing up in APA style manuscript. 
%
%\item[]{\textbf{\large{Lab Experience}}}
%

%\item \textbf{Research Assistant} {Fall 2012-Summer 2013 }\\
%Supervisor: Mike Whitson, iTracking Research, Inc.\\
%Assist in research design, methodologies, and conducting statistical analyes in R and SPSS. Assembled scripts for analyses, cleaned datafiles, constructed and ran SEM models in AMOS, composed graphs of analyses, and other duties as needed.     
%
%\item \textbf{Research Assistant} {Fall 2012-Summer 2013 }\\
%Supervisor: Dr. Andrew R. Gilpin, UNI\\
%Assist in conducting background research and conceptualizing a Monte Carlo simulation study that examines the effectiveness of basing sample size planning on the results of pilot studies.  
%
%\item \textbf{Research Assistant} {Spring 2012-Spring 2013 }\\
%Supervisor: Dr. Nicole Skaar, UNI\\
%Design and run confirmatory factor analyses in AMOS and R on a measure of risk behavior. Conduct IRT analysis of Iowa Youth Survey data. 
%
%\item \textbf{Lab Coordinator} Summer 2012-Fall 2012, Psychometrics and Personality Lab\\
%Supervisor: Dr. John E. Williams, UNI\\
%Responsibilities include coordinating lab related meetings and activities. Additionally, conduct meetings that teach research methods related to the current lab research. 
%


%%%%%%%%%%%%%%%%%%%%%%%%%%%%%%
\resheading{Teaching Experience}
%%%%%%%%%%%%%%%%%%%%%%%%%%%%%%
\begin{itemize} 
	\setlength{\topsep}{0pt}%
	\setlength{\leftmargin}{0.1in}%
	
	%%%%%%%%%% Short Version %%%%%%%%%%%%%%
	
	\item {\textbf{\large{Classes}}}
	
	\begin{center}
		\parbox{6.5in}{\textbf{Undergraduate Psych Statistics}}
		\parbox{6.5in}{PSY30100 - 03: Experimental Psychology I - Statistics}
		\parbox{6.5in}{Fall 2017, 2018, \& 2020, University of Notre Dame}
	\end{center}
	
	% 
	\begin{center}
		\parbox{6.5in}{\textbf{Graduate Regression}}
		\parbox{6.5in}{PSY60101 - 01: Quantitative Methods in Psychology II}
		\parbox{6.5in}{Spring 2018, 2019, 2020, 2021, \& 2022, University of Notre Dame}
	\end{center}
	
	\begin{center}
		\parbox{6.5in}{\textbf{Machine Learning for Social \& Behavioral Research}}
		\parbox{6.5in}{Joint Undergraduate \& Graduate}
		\parbox{6.5in}{Fall 2019 \& 2022, University of Notre Dame}
	\end{center}
	
	
	\item {\textbf{\large{Workshops}}}
	
	\begin{center}
		\parbox{6.5in}{\textbf{Advanced Machine Learning}}
		\parbox{6.5in}{Statistical Horizons. October 2021 \& May 2022}
		\parbox{6.5in}{Role: Instructor}
	\end{center}
	
\begin{center}
	\parbox{6.5in}{\textbf{Structural Equation Modeling in Longitudinal Research}}
	\parbox{6.5in}{APA Advanced Training Institutes. 2x: June 2020 \& July 2020}
	\parbox{6.5in}{Role: Instructor}
\end{center}

\begin{center}
	\parbox{6.5in}{\textbf{Exploratory Data Mining via SEARCH Strategies}}
	\parbox{6.5in}{8/15/2016-8/19/2016 \& 8/14/2017-8/18/2017}
	\parbox{6.5in}{University of Michigan, Ann Arbor, MI.}
	\parbox{6.5in}{Role: Lecturer and lab instructor.}
	%	\parbox{6.5in}{Lecture Topic: Structural Equation Model Trees}
	%	\parbox{6.5in}{Lab Topics: R, Regression, Clustering, Factor Analysis, Decision Trees \& Extensions, SEM Trees.}
\end{center}

%\item[]Exploratory Data Mining via SEARCH Strategies. 8/15/2016-8/19/2016, University of Michigan, Ann Arbor, MI.\\
%Role: Lecturer and lab instructor.
%
\begin{center}
	\parbox{6.5in}{\textbf{Big Data: Exploratory Data Mining in Behavioral Research}}
	\parbox{6.5in}{APA Advanced Training Institutes. 6/6/2016-6/10/2016 \& 6/5/2017-6/9/2017 }
	\parbox{6.5in}{Arizona State University, Tempe, AZ}
	\parbox{6.5in}{Role: Lecturer and lab instructor.}
	%	\parbox{6.5in}{Lecture Topic: Structural Equation Model Trees}
	%	\parbox{6.5in}{Lab Topics:  Decision Trees, Random Forests, Boosting, \& SEM Trees.}
\end{center}

%\item[]Big Data: Exploratory Data Mining in Behavioral Research, APA Advanced Training Institutes. 6/6/2016-6/10/2016, Arizona State University, Tempe, AZ. Role: Lecturer and lab instructor.
%
%\begin{center}
%	\parbox{6.5in}{\textbf{Structural Equation Modeling in Longitudinal Research}}
%	\parbox{6.5in}{APA Advanced Training Institutes. 6/6/2016-6/10/2016 \& 5/30/2017-6/3/2017}
%	\parbox{6.5in}{Arizona State University, Tempe, AZ}
%	\parbox{6.5in}{Role: Lab instructor.}
	%	\parbox{6.5in}{Lab Topics: Longitudinal Factor Analysis \& Alternate Change Score Models.}
%\end{center}

%\item[]Structural Equation Modeling in Longitudinal Research, APA Advanced Training Institutes. 6/6/2016-6/10/2016, Arizona State University, Tempe, AZ. Role: Lab instructor.
%
%\begin{center}
%	\parbox{6.5in}{\textbf{Exploratory Data Mining via SEARCH Strategies}}
%	\parbox{6.5in}{6/8/2015-6/12/2015, University of Michigan, Ann Arbor, MI}
%	\parbox{6.5in}{Role: Lab instructor.}
	%	\parbox{6.5in}{Lab Topics: R, Regression, Clustering, Factor Analysis, Decision Trees \& Extensions, SEM Trees.}
%\end{center}

%\item[]Exploratory Data Mining via SEARCH Strategies. 6/8/2015-6/12/2015, University of Michigan, Ann Arbor, MI. Role: Lab instructor.
%
%\begin{center}
%	\parbox{6.5in}{\textbf{Big Data: Exploratory Data Mining in Behavioral Research}}
%	\parbox{6.5in}{APA Advanced Training Institutes. 6/1/2015-6/5/2015, Arizona State University, Tempe, AZ}
%	\parbox{6.5in}{Role: Lab instructor.}
	%	\parbox{6.5in}{Lab Topics: Decision Trees \& Extensions, SEM Trees, Heuristic Search Algorithms.}
%\end{center}
%\item[]Big Data: Exploratory Data Mining in Behavioral Research, APA Advanced Training Institutes. 6/1/2015-6/5/2015, Arizona State University, Tempe, AZ. Role: Lab instructor.


%\item {\textbf{\large{Teaching Assistant}}}

%Fundamentals of Psychological Measurement, Fall 2014 (Graduate-Level: Dr. John J. McArdle, USC)\\
%History \& Systems, Spring 2013 (Undergraduate-Level: Dr. Andrew R. Gilpin, UNI)\\
%Psychology of Music, Spring 2013 (Undergraduate-Level: Dr. Carolyn Hildebrandt, UNI)\\
%Advanced Graduate Statistics, Fall 2012 (Graduate-Level: Dr. Andrew R. Gilpin, UNI)\\
%Cognitive \& Intellectual Assessment, Fall 2012 (Graduate-Level: Dr. John E. Williams, UNI)\\
%Honor's Introducton to Psychology, Fall 2012 (Undergraduate-Level: Dr. John E. Williams, UNI)\\
%Introduction to Psych Statistics, Spring 2012 (Undergraduate-Level: Dr. Andrew R. Gilpin, UNI)\\
%Memory \& Language, Fall 2011 (Undergraduate-Level: Dr. Jack Yates, UNI)\\
%
\end{itemize}



%%%%%%%%%%%%%%%%%%%%%%%%%%%%%%
\resheading{Professional Activities}
%%%%%%%%%%%%%%%%%%%%%%%%%%%%%%
\begin{itemize} 
	\setlength{\topsep}{0pt}%
	\setlength{\leftmargin}{0.1in}%
	
	\item {\textbf{\large{Editorial Boards}}}\\
	Psychological Methods\\
	Clinical Psychological Science\\
	Journal for Quantitative and Computational Methods in Behavioral Sciences (QCMB)
	
	
	%\item[] Team member of environmental-brain aging psychometrics working group for %the Environmental Determinants of Cognitive Aging R01, Summer 2013 - Spring 2014, %USC.
	
	%\item {\textbf{\large{Advanced Training}}}
	
	%\item[]Integrative Analysis of Longitudinal Studies of Aging and Dementia:
	%Enhancing Replication and Reproducibility, GSA 2015 Pre-Conference Workshop, Orlando, FL
	
	%\item[]Structural Equation Modeling in Longitudinal Research, APA Advanced Training Institutes. Summer 2014, Arizona State University, Tempe, AZ
	
	%\item[] Big Data: Exploratory Data Mining in Behavioral Research, APA Advanced Training Institutes. Summer 2014, Arizona State University, Tempe, AZ
	
	
	\item {\textbf{\large{Reviewing}}}
	
Journal of Consulting and Clinical Psychology; Structural Equation Modeling; Advances in Methods and Practices in Psychological Science; Multivariate Behavioral Research; Journal of Mathematical Psychology; Psychological Methods; Behavior Research Methods; Psychometrika; BMJ Open; Brain Sciences; JMIR Mental Health; Clinical Psychological Science; Addiction Research; Journal of Psychiatric Research; Suicide and Life-Threatening Behavior; British Journal of Clinical Psychology; Behavior Therapy; Journal of Clinical Child \& Adolescent Psychology; Archives of Suicide Research; Child Maltreatment; Perspectives on Psychological Science; Biological Psychiatry; Journal of Consulting and Clinical Psychology;  Addiction Research \& Theory\\
%21st IAGG World Congress of Gerontology and Geriatrics\\

\item {\textbf{\large{Grant Reviewing}}}

NIMH Urgent COVID-19 -- 02/2023\\
NIMH T32 -- 11/2020\\
NIMH R01 -- 07/2020\\
Swiss National Science Foundation -- 06/2019
	
	
	%\item {\textbf{\large{Affiliations}}}
	
	
	%\item{\large{International Society for Intelligence Research}}
	%\item{\large{Association for Psychological Science}}
	%Gerontological Society of America\\
	%\item{\large{APA Division 15. Educational Psychology}}
	%Psychometric Society\\
	
	
\end{itemize}


%%%%%%%%%%%%%%%%%%%%%%%%%%%%%%
%\resheading{Consultation}
%%%%%%%%%%%%%%%%%%%%%%%%%%%%%%
%begin{itemize} 
%\setlength{\topsep}{0pt}%
%\setlength{\leftmargin}{0.1in}

%\item \textbf{iTracking Research, Inc.} {Summer 2013-Fall 2013}\\
%Assisted in the methodological development of the analysis of Parkinson's Disease through the use of eyetracking. Developed scripts in R to organize the data for analysis as well as for the use of machine learning algorithms to predict outcomes. 
%
%end{itemize}






%%%%%%%%%%%%%%%%%%%%%%%%%%%%%%
%\resheading{Skills}
%%%%%%%%%%%%%%%%%%%%%%%%%%%%%%

%begin{itemize} 
%\setlength{\topsep}{0pt}%
%\setlength{\leftmargin}{0.1in}%
%
%\item \textbf{Statistical Software}\\
%R,  Mplus, Winsteps, flexMIRT, SAS, SPSS, Amos
%

%\item \textbf{Assessment Administration through Coursework} {Fall 2011-Spring 2012}\\
%Trained on the following measures: Minnesota Multiphasic Personality Inventory–	-Second Edition, Revised Form, and Adolescent Form; Personality Assessment Inventory–-Adult and Adolescent Forms; Structured Clinical Interview for DSM Disorders; Structured Clinical Interview for DSM Personality Disorders; Weschler’s Adult Intelligence Scale–-Fourth Edition; and Woodcock Johnson Achievement Test–-Third Edition.
%
%end{itemize}

%%%%%%%%%%%%%%%%%%%%%%%%%%%%%%
%\resheading{References}
%%%%%%%%%%%%%%%%%%%%%%%%%%%%%%

%   \begin{center}
%    \parbox{6.5in}{\textbf{Elizabeth Zelinski}, Ph.D., University of Southern California}
%   \parbox{6.5in}{Professor of Gerontology and Psychology}
%  \parbox{6.5in}{zelinski@usc.edu}
%  \parbox{6.5in}{Phone: (213) 740-4918}
%  \end{center}
%    \begin{center}
%    	\parbox{6.5in}{\textbf{Kevin J. Grimm}, Ph.D., Arizona State University}
%   	\parbox{6.5in}{Professor of Psychology}
%   	\parbox{6.5in}{kjgrimm@asu.edu}
%  	\parbox{6.5in}{Phone: (480) 965-5946 }
%  \end{center} 
%    \begin{center}
%   	\parbox{6.5in}{\textbf{John J. McArdle}, Ph.D., University of Southern California}
%  	\parbox{6.5in}{Professor of Psychology and Gerontology}
%  	\parbox{6.5in}{jack\_mcardle@hotmail.com}
%  	\parbox{6.5in}{Phone: (213) 740-2276 }
%  \end{center} 


%\begin{center}
%  \parbox{6.5in}{\textbf{Mike Whitson},  iTracking Research, Inc.}
%  \parbox{6.5in}{Vice President}
%  \parbox{6.5in}{mike@itrackingresearch.com}
%  \parbox{6.5in}{Phone: (319) 239-9008}
%  \end{center}
%\begin{center}
%  \parbox{6.5in}{\textbf{Nicole Skaar}, Ph.D., University of Northern Iowa}
%  \parbox{6.5in}{Assistant Professor of School Psychology}
%  \parbox{6.5in}{nicole.skaar@uni.edu}
%  \parbox{6.5in}{Phone: 319-273-7649}
%  \end{center}




\end{document}